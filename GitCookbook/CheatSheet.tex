%%%%%%%%%%%%%%%%%%%%%%%%%%%%%%%%%%%%%%%%%
% Git Cheat Sheet
% 
% This .tex file is downloaded form:
% https://github.com/HavocFiXer/GitLearning/blob/master/GitCookbook/CheatSheet.tex
%
% Author:
% HavocFiXer
%
%%%%%%%%%%%%%%%%%%%%%%%%%%%%%%%%%%%%%%%%%

%----------------------------------------------------------------------------------------
%	PACKAGES AND OTHER DOCUMENT CONFIGURATIONS
%----------------------------------------------------------------------------------------

\documentclass{article}

\usepackage{fancyhdr} % Required for custom headers
\usepackage{lastpage} % Required to determine the last page for the footer
\usepackage{extramarks} % Required for headers and footers
\usepackage[usenames,dvipsnames]{color} % Required for custom colors
\usepackage{graphicx} % Required to insert images
\usepackage{listings} % Required for insertion of code
\usepackage{courier} % Required for the courier font
\usepackage{lipsum} % Used for inserting dummy 'Lorem ipsum' text into the template
\usepackage{CJK}

% Margins
\topmargin=-0.45in
\evensidemargin=0in
\oddsidemargin=0in
\textwidth=6.5in
\textheight=9.0in
\headsep=0.25in

\linespread{1.1} % Line spacing

% Set up the header and footer
\pagestyle{fancy}
\lhead{\hmwkClass:\ \hmwkTitle} % Top left header
%\chead{\hmwkClass\ (\hmwkClassInstructor\ \hmwkClassTime): \hmwkTitle} % Top center head
\rhead{\firstxmark} % Top right header
\lfoot{\lastxmark} % Bottom left footer
\cfoot{} % Bottom center footer
\rfoot{Page\ \thepage\ of\ \protect\pageref{LastPage}} % Bottom right footer
\renewcommand\headrulewidth{0.4pt} % Size of the header rule
\renewcommand\footrulewidth{0.4pt} % Size of the footer rule

\setlength\parindent{0pt} % Removes all indentation from paragraphs

%----------------------------------------------------------------------------------------
%	CODE INCLUSION CONFIGURATION
%----------------------------------------------------------------------------------------

\definecolor{MyDarkGreen}{rgb}{0.0,0.4,0.0} % This is the color used for comments
\lstloadlanguages{bash} % Load Perl syntax for listings, for a list of other languages supported see: ftp://ftp.tex.ac.uk/tex-archive/macros/latex/contrib/listings/listings.pdf
\lstset{language=bash, % Use Perl in this example
        frame=single, % Single frame around code
		backgroundcolor=\color{White},
		basicstyle=\ttfamily, % Use small true type font
		%keywordstyle=\color{LimeGreen}\bf,
        keywordstyle=\color{LimeGreen}\bf, % Perl functions bold and blue
        keywordstyle=[2]\color{Cerulean}\bf, % Perl function arguments purple
        keywordstyle=[3]\color{Magenta}\bf,%\underbar, % Custom functions underlined and blue
		%otherkeywordsstyle=\color{LimeGreen},
		otherkeywords={-,^},
        identifierstyle=, % Nothing special about identifiers                                         
        commentstyle=\usefont{T1}{pcr}{m}{sl}\color{MyDarkGreen}\small, % Comments small dark green courier font
        stringstyle=\color{GreenYellow}, % Strings are purple
        showstringspaces=false, % Don't put marks in string spaces
        tabsize=4, % 5 spaces per tab
        %
        % Put standard Perl functions not included in the default language here
		morekeywords=[1]{global,HEAD},
        %
        % Put Perl function parameters here
        morekeywords=[2]{add},
        %
        % Put user defined functions here
        morekeywords=[3]{git},
       	%
        morecomment=[l][\color{Blue}]{...}, % Line continuation (...) like blue comment
        numbers=left, % Line numbers on left
        firstnumber=1, % Line numbers start with line 1
        numberstyle=\tiny\color{Blue}, % Line numbers are blue and small
        stepnumber=5 % Line numbers go in steps of 5
}

% Creates a new command to include a perl script, the first parameter is the filename of the script (without .pl), the second parameter is the caption
\newcommand{\sqlscript}[2]{
\begin{itemize}
\item[]\lstinputlisting[caption=#2,label=#1]{#1.sql}
\end{itemize}
}

%----------------------------------------------------------------------------------------
%	DOCUMENT STRUCTURE COMMANDS
%	Skip this unless you know what you're doing
%----------------------------------------------------------------------------------------

% Header and footer for when a page split occurs within a problem environment
\newcommand{\enterProblemHeader}[1]{
\nobreak\extramarks{#1}{#1 continued on next page\ldots}\nobreak
\nobreak\extramarks{#1 (continued)}{#1 continued on next page\ldots}\nobreak
}

% Header and footer for when a page split occurs between problem environments
\newcommand{\exitProblemHeader}[1]{
\nobreak\extramarks{#1 (continued)}{#1 continued on next page\ldots}\nobreak
\nobreak\extramarks{#1}{}\nobreak
}

\setcounter{secnumdepth}{0} % Removes default section numbers
\newcounter{homeworkProblemCounter} % Creates a counter to keep track of the number of problems

\newcommand{\homeworkProblemName}{}
\newenvironment{homeworkProblem}[1][Chapter \arabic{homeworkProblemCounter}]{ % Makes a new environment called homeworkProblem which takes 1 argument (custom name) but the default is "Problem #"
\stepcounter{homeworkProblemCounter} % Increase counter for number of problems
\renewcommand{\homeworkProblemName}{#1} % Assign \homeworkProblemName the name of the problem
\section{\homeworkProblemName} % Make a section in the document with the custom problem count
\enterProblemHeader{\homeworkProblemName} % Header and footer within the environment
}{
\exitProblemHeader{\homeworkProblemName} % Header and footer after the environment
}

\newcommand{\problemAnswer}[1]{ % Defines the problem answer command with the content as the only argument
\noindent\framebox[\columnwidth][c]{\begin{minipage}{0.98\columnwidth}#1\end{minipage}} % Makes the box around the problem answer and puts the content inside
}

\newcommand{\homeworkSectionName}{}
\newenvironment{homeworkSection}[1]{ % New environment for sections within homework problems, takes 1 argument - the name of the section
\renewcommand{\homeworkSectionName}{#1} % Assign \homeworkSectionName to the name of the section from the environment argument
\subsection{\homeworkSectionName} % Make a subsection with the custom name of the subsection
\enterProblemHeader{\homeworkProblemName\ [\homeworkSectionName]} % Header and footer within the environment
}{
\enterProblemHeader{\homeworkProblemName} % Header and footer after the environment
}

%----------------------------------------------------------------------------------------
%	NAME AND CLASS SECTION
%----------------------------------------------------------------------------------------

\newcommand{\hmwkTitle}{CheatSheet} % Assignment title
\newcommand{\hmwkDueDate}{05,\ 02,\ 2016} % Due date
\newcommand{\hmwkClass}{Git} % Course/class
%\newcommand{\hmwkClassTime}{10:30am} % Class/lecture time
%\newcommand{\hmwkClassInstructor}{Jones} % Teacher/lecturer
\newcommand{\hmwkStatement}{With salute to Mr. Linus Torvalds} % Teacher/lecturer
\newcommand{\hmwkAuthorName}{Hongfei Xue} % Your name
\newcommand{\hmwkUBIT}{Email: hongfeixue222@gmail.com} % Your name


%----------------------------------------------------------------------------------------
%	TITLE PAGE
%----------------------------------------------------------------------------------------

\title{
\vspace{2in}
\textmd{\textbf{\hmwkClass:\ \hmwkTitle}}\\
\normalsize\vspace{0.1in}\small{Last\ modification\ on\ \hmwkDueDate}\\
\vspace{1in}\large{\textit{\hmwkStatement}}
\vspace{2in}
}

\author{\textbf{\hmwkAuthorName}\\ \textbf{\hmwkUBIT}}
\date{} % Insert date here if you want it to appear below your name

%----------------------------------------------------------------------------------------

\begin{document}

\maketitle

%----------------------------------------------------------------------------------------
%	TABLE OF CONTENTS
%----------------------------------------------------------------------------------------

%\setcounter{tocdepth}{1} % Uncomment this line if you don't want subsections listed in the ToC

\newpage
\begin{CJK*}{GBK}{song}
\tableofcontents
\end{CJK*}
\newpage

%----------------------------------------------------------------------------------------
%	PROBLEM 1
%----------------------------------------------------------------------------------------

% To have just one problem per page, simply put a \clearpage after each problem

\begin{homeworkProblem}
\begin{CJK*}{GBK}{song}
\subsection{���}
Git��Ŀǰ���������Ƚ��ķֲ�ʽ�汾����ϵͳ��
\end{CJK*}

\subsubsection{View SEC}
\begin{lstlisting}
>>git add --global HEAD^ "status.txt"
\end{lstlisting}

\subsection{Q1.1}
SQL solution:
\ref{Q1.1}
\sqlscript{Q1.1}{}
Algebra solution:
\begin{equation}
	RESULT(SNO):=\pi_{CNO}(\sigma_{DEPT="MTH"}(STUDENT \mathop{\bowtie}_{CNO=CNO} ENROLL))
\end{equation}

\subsection{Q1.2}
SQL solution:
\ref{Q1.2}
\sqlscript{Q1.2}{}
Algebra solution:
\begin{equation}
	SE(GRADE, NAME) := \pi_{GRADE, NAME}(STUDENT \mathop{\bowtie}_{SNO=SNO} ENROLL)
\end{equation}
\begin{equation}
	RESULT(SNO) := \pi_{NAME}(SE) - \pi_{NAME}(\sigma_{GRADE<3.66}(SE))
\end{equation}

\subsection{Q1.3}
SQL solution:
\ref{Q1.3}
\sqlscript{Q1.3}{}
Algebra solution:
\begin{equation}
	S(SNO,GRADE):=\pi_{SNO,GRADE}(\sigma_{DEPT='CSE'}(COURSE\mathop{\bowtie}_{CNO=CNO}ENROLL))
\end{equation}
\begin{equation}
	M(SNO) := \pi_{SNO}(S)-\pi_{SNO}(S(SNO1, GRADE1) \mathop{\bowtie}_{GRADE1>GRADE2} S(SNO2, GRADE2))
\end{equation}
\begin{equation}
	RESULT(DEPT) := \pi_{DEPT}(M \mathop{\bowtie}_{SNO=SNO} STUDENT)
\end{equation}

\end{homeworkProblem}

%----------------------------------------------------------------------------------------
%	PROBLEM 2
%----------------------------------------------------------------------------------------

\begin{homeworkProblem}
\subsection{View Used for This Problem}

\subsubsection{View SEC}
\ref{SEC2}
\sqlscript{SEC2}{}%to add description
\subsubsection{View SE}
\ref{SE2}
\sqlscript{SE2}{}%to add description
\subsubsection{View ALLS}
\ref{ALLS}
\sqlscript{ALLS}{}%to add description
\subsubsection{View NOTS}
\ref{NOTS}
\sqlscript{NOTS}{}%to add description
\subsubsection{View NGB}
\ref{NGB}
\sqlscript{NGB}{}%to add description

\subsection{Q2.1}
SQL solution:
\ref{Q2.1}
\sqlscript{Q2.1}{} 
\subsection{Q2.2}
SQL solution:
\ref{Q2.2}
\sqlscript{Q2.2}{}

\subsection{Q2.3}
SQL solution:
\ref{Q2.3}
\sqlscript{Q2.3}{}

\end{homeworkProblem}

%----------------------------------------------------------------------------------------
\begin{homeworkProblem}

\subsection{Q3.1}
\textbf{Explanation:} Find the numbers of all the students who enrolled only 1 course.\\
Algebra solution:
\begin{equation}
	G(SNO):=\pi_{SNO}(ENROLL \mathop{\bowtie}_{SNO=SNO and CNO<>CNO} ENROLL)
\end{equation}
\begin{equation}
	RESULT(SNO):=\pi_{SNO}(ENROLL)-\pi_{SNO}(G)
\end{equation}

\subsection{Q3.2}
\textbf{Explanation:} Find the numbers of all the students who have enrolled courses.\\
\begin{equation}
	RESULT(SNO):= \pi_{SNO}(SUTDENT \mathop{\bowtie}_{SNO=SNO} ENROLL)
\end{equation}

\subsection{Q3.3}
\textbf{Explanation:} Find the numbers of all the students who have enrolled courses.

\end{homeworkProblem}
%----------------------------------------------------------------------------------------
\begin{homeworkProblem}
\subsection{View Used for This Problem}

\subsubsection{View CNP}
\ref{CNP}
\sqlscript{CNP}{}%to add description
\subsubsection{View CNA}
\ref{CNA}
\sqlscript{CNA}{}%to add description

\subsection{Q4.1}
SQL solution:
\ref{Q4.1}
\sqlscript{Q4.1}{}

\subsection{Q4.2}
SQL solution:
\ref{Q4.2}
\sqlscript{Q4.2}{}

\subsection{Q0}
\textbf{Note that in my solution I didn't use new claus of SQL but some basic calculations to solve this problem.}\\
SQL solution:
\ref{Q0}
\sqlscript{Q0}{}

\end{homeworkProblem}
%----------------------------------------------------------------------------------------


\end{document}
